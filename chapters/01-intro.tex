\chapter{Introduction}\label{ch:introduction}

\section{Motivation}

Many applications today perform real-time monitoring and analytics on
potentially unbounded streams of data. Examples of this include analytics for
web applications and social networks, online processing of sensor data, 
or monitoring and networking in datacenters. As the pool of data sources
in this environment grows larger, and the computations become more complex,
the need for stream management systems and expressive programming models arises.

One such programming model is the \emph{timely dataflow} model, first implemented
and introduced in the Naiad system \cite{naiad}. The timely dataflow model
is a programming model for a wide range of data-parallel algorithms, and it has
been demonstrated to perform well for graph and stream processing, also in the
form of incremental and differential computations \cite{differential}.
Timely dataflow supports stateful operators and iterative computations, while
achieving high throughput and low-latency responses thanks to a decentralized
coordination mechanism for global progress tracking.

In this thesis, we investigate the architecture and implementation of a
management system for running timely dataflow computations. Our work
builds upon the Rust implementation of the timely dataflow model,
simply called \textquote{Timely Dataflow} \cite{timely}.

\section{Contribution}

Being an embeddable library, the provided run-time in the current Rust implementation
only concerns itself with executing and scheduling a single dataflow computation.
It does support the execution in multiple processes distributed across a
static set of machines, launching and deployment of the resulting application
in the cluster however has to be done manually by the user. External inputs are
fed into the dataflow graph by custom client code.

The goal of this thesis is the design and implementation of a system which
provides the following features:

- launching and managing multiple concurrent dataflow programs
- on a dynamic set of machines
- providing mechanisms for dataflow computations to share data stream
- providing mechanisms for system state introspection

\section{Outline}

In chapter \ref{ch:background}, we provide a brief overview of
the Timely Dataflow library, its conceptual model and its implementation. The
system designed and implemented as part of this thesis is introduced and
discussed in detail in the chapters \ref{ch:design} \& \ref{ch:impl}. As our
system allows the dynamic composition of dataflow programs, we evaluate the
overhead of this feature on a realistic workload in chapter \ref{ch:evaluation}.
A survey of related work is presented in chapter \ref{ch:related}, and chapter
\ref{ch:future} closes with future work and conclusions.
